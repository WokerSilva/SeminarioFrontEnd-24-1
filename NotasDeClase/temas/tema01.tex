%%%%%%%%%%%%%%%%%%%%%%%%%%%%%%%
%%%%%%%%%%%%%%%%%%%%%%%%%%%
%%%%%%%%%%%%%%%%%%%%%%%
\section{Intrdoucción al desarrollo web}
%%%%%%%%%%%%%%%%%%%%%%%
%%%%%%%%%%%%%%%%%%%%%%%%%%%
%%%%%%%%%%%%%%%%%%%%%%%%%%%%%%%

%%%%%%%%%%%%%%%%%%%%%%%%%%%
%%%%%%%%%%%%%%%%%%%%%%%
\subsection{Internet y la web}
%%%%%%%%%%%%%%%%%%%%%%%
%%%%%%%%%%%%%%%%%%%%%%%%%%%
\begin{itemize}
    \item El internet (o, también, la internet) es un conjunto descentralizado
     de redes de comunicación interconectadas que utilizan la familia de 
     protocolos TCP/IP
    \item Ofrece una serie de servicios, siendo la World Wide Web 
    (o Web o WWW) el de más éxito y conocido por todos.
\end{itemize}

La Web se desarrolló entre marzo de 1989 y diciembre de 1990 por el inglés Sir Tim Berners-Lee
con la ayuda del belga Robert Cailliau mientras trabajaban en el CERN en Ginebra, 
Suiza. La Web nace a partir de la creación de los siguientes estándares:

\begin{itemize}
    \item \textbf{HTML} (Hypertext Markup Language), se utiliza para definir la estructura
    \item \textbf{URL}  (Uniform Resource Locator), se utiliza para referenciar recursos en la Web de forma universal.
    \item \textbf{HTTP} (HyperText Transfer Protocol), es el protocolo de comunicación que permite las transferencias de información en la World Wide Web.
\end{itemize}

Cuando hablamos de Desarrollo Web hablamos necesariamente de \textbf{Aplicaciones Web} que
es un posible producto resultante de esa tarea de desarrollo. Una aplicación web o, de forma abreviada, webapp, es una aplicación informática que un
usuario puede utilizar accediendo a través de internet a un \textbf{servidor} web, y para ello hace
uso de un \textbf{cliente} web o navegador.


%%%%%%%%%%%%%%%%%%%%%%%%%%%
%%%%%%%%%%%%%%%%%%%%%%%
\subsection{Tipos de comunicación Cliente Servidor}
%%%%%%%%%%%%%%%%%%%%%%%
%%%%%%%%%%%%%%%%%%%%%%%%%%%

\begin{center}
\begin{minipage}[t]{0.45\textwidth}
    \begin{center}
        Solicit/Response    

        \begin{tabular}{c c c}
                    & $\Rightarrow_{1}$ &  \\ 
            Cliente &               & Servidor \\
                    & $\Leftarrow_{2}$ &  \\ 
        \end{tabular}
    \end{center}

    Funcionamiento: El cliente envía una solicitud al servidor, y el servidor responde con la información solicitada.
    Ejemplo: Un navegador web (cliente) solicita una página web al servidor web. El servidor procesa la solicitud y devuelve la página web al navegador.
\end{minipage}
\hspace{15mm}
\begin{minipage}[t]{0.45\textwidth}
    \begin{center}
        Request/Response    

        \begin{tabular}{c c c}
                    & $\Leftarrow_{1}$ &  \\                 
            Cliente &               & Servidor \\
                    & $\Rightarrow_{2}$ &  \\ 
        
        \end{tabular}
    \end{center}

    Funcionamiento: El cliente hace una petición al servidor y recibe una respuesta. 
    Ejemplo: Una aplicación de correo electrónico (cliente) envía una petición al servidor de correo para descargar nuevos mensajes. El servidor responde con los mensajes nuevos.
\end{minipage}
\end{center}

\vspace{3mm}

\begin{center}
    \begin{minipage}[t]{0.45\textwidth}
        \begin{center}
            Notification        
    
            \begin{tabular}{c c c}
                        & $\Leftarrow_{1}$ &  \\                 
                Cliente &               & Servidor                    
            \end{tabular}
        \end{center}

        Funcionamiento: El servidor envía notificaciones o actualizaciones al cliente sin esperar una respuesta inmediata. 
        Ejemplo: WhatsApp, puede enviar notificaciones al cliente cuando llega un nuevo mensaje, incluso si el cliente no ha solicitado activamente los nuevos mensajes.
    \end{minipage}
    \hspace{15mm}
    \begin{minipage}[t]{0.45\textwidth}
        \begin{center}
            One-Way
            
            \begin{tabular}{c c c}
                        & $\Rightarrow_{1}$ &  \\ 
                Cliente &               & Servidor \\            
            \end{tabular}
        \end{center}

        Funcionamiento: Comunicación unidireccional, cliente envía una solicitud al servidor sin esperar una respuesta inmediata o sin necesidad de recibir una respuesta
        Ejemplo: En un sistema de telemetría de sensores, un sensor puede enviar datos al servidor para informar sobre las lecturas sin necesidad de que el servidor responda.
    \end{minipage}
\end{center}
    
%%%%%%%%%%%%%%%%%%%%%%%%%%%
%%%%%%%%%%%%%%%%%%%%%%%
\subsection*{Frontend y Backend}
%%%%%%%%%%%%%%%%%%%%%%%
%%%%%%%%%%%%%%%%%%%%%%%%%%%
\begin{center}
\begin{minipage}[position]{0.40\textwidth}
    En líneas generales, los
    desarrolladores frontend se
    encargan de diseñar y construir
    los elementos con los que el
    público tendrá contacto. Aquello
    incluye los botones, menús,
    páginas, enlaces, gráficos y otros
    componentes de una página o
    aplicación.
\end{minipage}
\hspace{5mm}
\begin{minipage}[position]{0.40\textwidth}
    El backend consiste del servidor
    que provee la información que se
    solicita, la aplicación que se
    encarga de canalizarla y la base
    de datos que organiza la
    información. Por ejemplo, cuando
    un cliente busca zapatos en un
    sitio web, este interactúa con el
    frontend.
\end{minipage}
\end{center}

%%%%%%%%%%%%%%%%%%%%%%%%%%%%%%%
%%%%%%%%%%%%%%%%%%%%%%%%%%%
%%%%%%%%%%%%%%%%%%%%%%%
\section{Intrdoucción al Frontend}
%%%%%%%%%%%%%%%%%%%%%%%
%%%%%%%%%%%%%%%%%%%%%%%%%%%
%%%%%%%%%%%%%%%%%%%%%%%%%%%%%%%

\begin{quote}
El desarrollo frontend es el proceso de creación de componentes interactivos para 
el usuario final de una aplicación web. Las interfaces de usuario, los botones, los
datos ingresados por el usuario, las páginas web y las funciones de experiencia 
del usuario (UX) son ejemplos de desarrollo frontend.
\end{quote}

%%%%%%%%%%%%%%%%%%%%%%%%%%%%%%
%%%%%%%%%%%%%%%%%%%%%%%%%%%
\subsection*{Objetivos}
%%%%%%%%%%%%%%%%%%%%%%%%%%%
%%%%%%%%%%%%%%%%%%%%%%%%%%%%%%
\begin{itemize}
    \item Organizar y presentar la información, manteniéndola relevante,
     fácil de encontrar y acceder a ella.
    \item Lidiar con los múltiples dispositivos desde los cuales se 
     pueden acceder a la información.    
\end{itemize}

%%%%%%%%%%%%%%%%%%%%%%%%%%%%%%
%%%%%%%%%%%%%%%%%%%%%%%%%%%
\subsection*{Ventajas del desarrollo web frontend}
%%%%%%%%%%%%%%%%%%%%%%%%%%%
%%%%%%%%%%%%%%%%%%%%%%%%%%%%%%
\begin{itemize}
    \item Proporciona un desarrollo rápido gracias a todos los
     frameworks e innovaciones modernas disponibles.
    \item Las herramientas y técnicas son fáciles de aprender. La mayor
    parte del desarrollo frontend se limita a las tres tecnologías principales que son HTML, CSS y JavaScript. 
\end{itemize}

%%%%%%%%%%%%%%%%%%%%%%%%%%%
%%%%%%%%%%%%%%%%%%%%%%%
\subsection*{Desventajas del desarrollo web frontend}
%%%%%%%%%%%%%%%%%%%%%%%
%%%%%%%%%%%%%%%%%%%%%%%%%%%
\begin{itemize}
    \item No importa cuán grande o pequeño sea un sitio web, la
     personalización es una parte esencial. Esto nos lleva a una base de código usualmente grande.
    \item En comparación con los lenguajes utilizados en el backend
     como PHP y Java, que existen desde hace bastante tiempo, JavaScript es bastante nuevo.
    \item Se lanzan versiones nuevas de bibliotecas y de los frameworks rápidamente.
     Lidiar con actualizaciones frecuentes es complicado, con cada nueva version,
      existe una mayor posibilidad de cometer errores y que funcionalidad previa no se encuentre soportada.
\end{itemize}