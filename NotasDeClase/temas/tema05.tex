%%%%%%%%%%%%%%%%%%%%%%%%%%%%%%%
%%%%%%%%%%%%%%%%%%%%%%%%%%%
%%%%%%%%%%%%%%%%%%%%%%%
\section{GIT}
%%%%%%%%%%%%%%%%%%%%%%%
%%%%%%%%%%%%%%%%%%%%%%%%%%%
%%%%%%%%%%%%%%%%%%%%%%%%%%%%%%%


Git es un sistema de control de versiones distribuido y gratuito.
\begin{itemize}
    \item Permite realizar un seguimiento de los cambios de código.
    \item Permite crear nuevas ramas para guardar diferentes versiones del código.
    \item Cada rama contiene un conjunto de nuevas características que se están desarrollando.
\end{itemize}

\subsection{Áreas de trabajo}

En Git, el proceso de trabajo se organiza en tres áreas clave: el directorio de trabajo (working
 directory), el área de preparación (staging area), y el repositorio (repository). Cada área 
 cumple un papel específico en la gestión y seguimiento de los cambios en el código.

\begin{itemize}
    \item Working directory
    \begin{itemize}
        \item En esta área, se lleva a cabo todo el trabajo activo de creación, modificación, eliminación y organización de archivos y código.
        \item Los cambios realizados en el working directory no se registran automáticamente en el repositorio, lo que brinda flexibilidad para experimentar y probar cambios antes de confirmarlos.
    \end{itemize}
    \begin{lstlisting}[style=bashstyle]
        # Crear un nuevo archivo en el directorio de trabajo
        touch nuevo_archivo.txt
        
        # Modificar el archivo existente
        nano nuevo_archivo.txt
    \end{lstlisting}

    \item Staging area
    \begin{itemize}
        \item En esta etapa, se seleccionan y preparan los cambios específicos del working directory que se desean incluir en el próximo commit.
        \item La staging area actúa como un espacio intermedio donde se pueden revisar y ajustar los cambios antes de confirmarlos en el repositorio.
    \end{itemize}
    \begin{lstlisting}[style=bashstyle]        
        # Agregar cambios al area de preparacion
        git add nuevo_archivo.txt        
    \end{lstlisting}


    \item Repository
    \begin{itemize}
        \item Aquí es donde se almacenan permanentemente los cambios confirmados, incluyendo código, imágenes, logs, y demás.
        \item Cada confirmación (commit) en el repositorio tiene un historial que permite rastrear el progreso del proyecto y revertir a versiones anteriores si es necesario.
    \end{itemize}
    \begin{lstlisting}[style=bashstyle]        
        # Confirmar los cambios en el repositorio
        git commit -m "Actualizar nuevo archivo"
    \end{lstlisting}
\end{itemize}

\subsection{Ramas}

En Git, las ramas son un concepto esencial que permite crear entornos de trabajo independientes 
dentro de un proyecto. Se pueden interpretar como solicitudes para establecer nuevos contextos 
o versiones del proyecto, cada una enfocada en el desarrollo de conjuntos específicos de características.

Cada rama representa una línea de desarrollo separada, con su propio conjunto de 
cambios y actualizaciones. Este enfoque brinda la flexibilidad de trabajar en nuevas 
funcionalidades o correcciones sin afectar directamente la rama principal del proyecto. 
La rama predeterminada en Git se denomina \textit{master}, y es la línea base desde la cual 
se pueden ramificar otras versiones del código.

\subsubsection*{Flujo de trabajo}
\begin{enumerate}
    \item Crear la rama 
    \begin{lstlisting}[style=bashstyle]
        git branch nueva-funcionalidad-rama
    \end{lstlisting}

    \item Cambiar a la nueva rama 
    \begin{lstlisting}[style=bashstyle]
        git checkout nueva-funcionalidad-rama
    \end{lstlisting}

    \item Hacer modificaciones a los archivos de las ramas

    \item Agregar y confirmar cambios en la nueva rama
    \begin{lstlisting}[style=bashstyle]
        git add .
        git commit -m "Agregar nueva funcionalidad"
    \end{lstlisting}

    \item Volver a la rama principal (o de donde salio la rama)
    \begin{lstlisting}[style=bashstyle]
        git branch master
    \end{lstlisting}

    \item Integrar la nueva funcionalidad en la rama principal
    \begin{lstlisting}[style=bashstyle]
        git merge nueva-funcionalidad-rama
    \end{lstlisting}
\end{enumerate}

\subsection{Convencion de Commits}

Las convenciones de commits son prácticas que ayudan a estandarizar y estructurar los mensajes de 
confirmación (commits) en un repositorio Git. Estas convenciones facilitan la comprensión del 
historial del proyecto y la colaboración entre desarrolladores. 
Algunas de las convenciones comunes incluyen:\\

\begin{itemize}
    \item Se utiliza un prefijo para indicar el propósito del commit. 
    \begin{itemize}
        \item feat: Nueva característica.
        \begin{lstlisting}[style=bashstyle]
            git commit -m "feat: Agregar funcionalidad de autenticacion"
        \end{lstlisting}
        \item fix: Corrección de un error.
        \item docs: Cambios en la documentación.
        \item style: Mejoras en el formato/código (sin cambios en la lógica).
        \item chore: Tareas de mantenimiento, refactorización, etc.
    \end{itemize}
\end{itemize}