%%%%%%%%%%%%%%%%%%%%%%%%%%%%%%%
%%%%%%%%%%%%%%%%%%%%%%%%%%%
%%%%%%%%%%%%%%%%%%%%%%%
\section{Introducción a JavaScript}
%%%%%%%%%%%%%%%%%%%%%%%
%%%%%%%%%%%%%%%%%%%%%%%%%%%
%%%%%%%%%%%%%%%%%%%%%%%%%%%%%%%


%%%%%%%%%%%%%%%%%%%%%%%%%%%
%%%%%%%%%%%%%%%%%%%%%%%
\subsection{¿Qué es JavaScript?}
%%%%%%%%%%%%%%%%%%%%%%%
%%%%%%%%%%%%%%%%%%%%%%%%%%%

JavaScript es un lenguaje de programación interpretado basado en texto que nos permite agregar funcionalidad avanzada a cualquier páginaweb.

JavaScriptes increíblementeligero y se usa principalmenteparacrearscriptsen páginas web. También se utiliza para crear aplicacionesweb que interactúan conel cliente sin recargarla página cada vez.

%%%%%%%%%%%%%%%%%%%%%%%%%%%
%%%%%%%%%%%%%%%%%%%%%%%
\subsection{Motores Javascript}
%%%%%%%%%%%%%%%%%%%%%%%
%%%%%%%%%%%%%%%%%%%%%%%%%%%

Siemprequese ejecutaun programaJavaScript dentro del navegadorweb, el código es recibido por el motor del navegadory luego el motor lo ejecuta.
El motor JavaScript tiene diferentes partes que nos ayudan en la ejecución del código.
\begin{itemize}
    \item Parser
    \item AST
    \item MachineCodeConversion
    \item MachineCode
\end{itemize}
 
%%%%%%%%%%%%%%%%%%%%%%%%%%%
%%%%%%%%%%%%%%%%%%%%%%%
\subsubsection*{Fases de ejecución del códio JS}
%%%%%%%%%%%%%%%%%%%%%%%
%%%%%%%%%%%%%%%%%%%%%%%%%%%

\begin{enumerate}
    \item El parser convierte los archivos de código fuente en un árbol de sintaxis abstracta (AST).
    \item El Árbol de sintaxis se transformaa bytecode: el intérprete de V8, Ignition, genera bytecode a partir del árbol de sintaxis (este paso existe desde 2017)
    \item El bytecode genera código de máquina: el compilador de V8, TurboFan, genera un grafo a partir del bytecode, reemplazando secciones de bytecode con código máquina altamente optimizado.
\end{enumerate}


%%%%%%%%%%%%%%%%%%%%%%%%%%%
%%%%%%%%%%%%%%%%%%%%%%%
\subsection{DOM y Javascript}
%%%%%%%%%%%%%%%%%%%%%%%
%%%%%%%%%%%%%%%%%%%%%%%%%%%

\textbf{DOM} significa \textit{Document Object Model}. El DOM es multiplataforma, independiente del lenguaje y que opera mediante la construcción de una estructura de árbol a partir del contenido HTML

La API del DOMse utiliza para cambiarla interfaz que el usuariove con JavaScript. Esto se logra alterando dinamicamente HTMLy el CSS


%%%%%%%%%%%%%%%%%%%%%%%%%%%%%%%
%%%%%%%%%%%%%%%%%%%%%%%%%%%
%%%%%%%%%%%%%%%%%%%%%%%
\subsection{Elementos de JavaScript}
%%%%%%%%%%%%%%%%%%%%%%%
%%%%%%%%%%%%%%%%%%%%%%%%%%%
%%%%%%%%%%%%%%%%%%%%%%%%%%%%%%%

%%%%%%%%%%%%%%%%%%%%%%%%%%%
%%%%%%%%%%%%%%%%%%%%%%%
\subsubsection*{Variables de JavaScript}
%%%%%%%%%%%%%%%%%%%%%%%
%%%%%%%%%%%%%%%%%%%%%%%%%%%

\begin{itemize}
    \item Un identificador, el nombre de una variable, debe comenzar con una letra ASCII mayúscula o minúscula, un signo de dólar  $ (\$) $ o un guión bajo $(_)$.    
    \item Se puede usar números en un identificador, pero no como primer carácter.
    \item No puede incluir espacios.
    \item No se pueden utilizar palabras reservadas para los identificadores.
\end{itemize}

%%%%%%%%%%%%%%%%%%%%%%%
\subsubsection*{Variables de JavaScript}
%%%%%%%%%%%%%%%%%%%%%%%

\begin{center}
    \begin{tabular}{ccccc}
        abstrac      & arguments & await    & boolean    & break     \\
        byte         & case      & catch    & char       & class     \\
        const        & continue  & debugger & default    & delate    \\
        do           & double    & else     & enum       & eval      \\
        export       & extends   & FALSE    & final      & finally   \\
        float        & for       & function & goto       & if        \\
        implements   & import    & in       & instanceof & int       \\
        interface    & let       & long     & native     & new       \\
        null         & package   & private  & protected  & public    \\
        return       & short     & static   & super      & switch    \\
        synchronized & this      & throw    & throws     & transient \\
        TRUE         & try       & typeof   & var        & void      \\
        volatile     & while     & with     & yield      &          
        \end{tabular}\end{center}

%%%%%%%%%%%%%%%%%%%%%%%
\subsubsection*{Tipos de Datos JS}
%%%%%%%%%%%%%%%%%%%%%%%

\begin{center}
    \begin{minipage}{5cm}
        \begin{tabular}{|c|}
            \hline
            primitive \\ \hline
            Boolean \\
            Null \\
            undefined \\
            Number \\
            String \\
            Symbol \\ \hline        
        \end{tabular}    
    \end{minipage}
    \begin{minipage}[c]{5cm}
        \begin{tabular}{|c|}
            \hline
            \multicolumn{1}{|c|}{object} \\ \hline             
            array \\
            Object \\
            Function \\
            RegEx \\
            Date \\ \hline
        \end{tabular}
    \end{minipage}
\end{center}