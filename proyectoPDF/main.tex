\documentclass[a4paper,12pt]{article} 
\usepackage[utf8]{inputenc} % Acentos válidos sin problemas
\usepackage[spanish]{babel} % Idioma
%-----------------------------------INICIO DE PACKETES------------------/
%----------------------------------------------------------------------/|
\usepackage{amsmath}   % Matemáticas: Comandos extras(cajas ecuaciones) |
\usepackage{amssymb}   % Matemáticas: Símbolos matemáticos              |
\usepackage{datetime}  % Agregar fechas                                 |
\usepackage{graphicx}  % Insertar Imágenes                              |
%\usepackage{biblatex} % Bibliografía                                   |
\usepackage{multicol}  % Creación de tablas                             |
\usepackage{longtable} % Tablas más largas                              |
\usepackage{xcolor}    % Permite cambiar colores del texto              |
\usepackage{tcolorbox} % Cajas de color                                 |
\usepackage{setspace}  % Usar espacios                                  |
\usepackage{fancyhdr}  % Para agregar encabezado y pie de página        |
\usepackage{float}     % Flotantes                                      |
\usepackage{soul}      % "Efectos" en palabras                          |
\usepackage{hyperref}  % Para usar hipervínculos                        |
\usepackage{caption}   % Utilizar las referencias                       |
\usepackage{subcaption} % Poder usar subfiguras                         |
\usepackage{multirow}  % Nos permite modificar tablas                   |
\usepackage{array}     % Permite utilizar los valores para multicolumn  |
\usepackage{booktabs}  % Permite modificar tablas                       |
\usepackage{diagbox}   % Diagonales para las tablas                     |
\usepackage{colortbl}  % Color para tablas                              |
\usepackage{listings}  % Escribir código                                |
\usepackage{mathtools} % SIMBOLO :=                                     |
\usepackage{enumitem}  % Modificar items de Listas                      |
\usepackage{tikz}      %                                                |
\usepackage{lipsum}    % for auto generating text                       |
\usepackage{afterpage} % for "\afterpage"s                              |
\usepackage{pagecolor} % With option pagecolor={somecolor or none}|     | 
%\usepackage{glossaries} %                                              |        
%----------------------------------------------------------------------\|
%-----------------------------------FIN--- DE PACKETES------------------\
%\usepackage{newtxtext,newtxmath}
%\usepackage{bm}
%--------------------------------/
%-------------------------------/
\usepackage[                 %   |
  headheight=15pt,  %            |
  letterpaper,  % Tipo de pag.   |
  left =1.5cm,  %  < 1 >         |
  right =1.5cm, %  < 1 >         | MARGENES DE LA PAGINA
  top =2cm,     %  < 1.5 >       |
  bottom =1.5cm %  < 1.5 >       |
]{geometry}     %                |
%-------------------------------\
%--------------------------------\

%----------------------------------------------------------------------/
%-------------------Encabezado y Pie de Pagina-----------------------/ |
%--------------------------------------------------------------------\ |
\usepackage{fancyhdr} %                                                |
\usepackage{lastpage} %                                                |
%\fancyhf{}           %                                                |
\pagestyle{fancy}     %                                                |
\fancyhead[L]{{2024-1}} %                                              |
\fancyhead[C]{}                      
\fancyhead[R]{}

\fancyfoot[L]{} 
\fancyfoot[C]{\thepage}
\fancyfoot[R]{} 

\renewcommand{\headrulewidth}{1.5pt} %                                 |
\renewcommand{\footrulewidth}{1.5pt} %                                 | 
%--------------------------------------------------------------------\ |
%-------------------Encabezado y Pie de Pagina-----------------------/ |
%----------------------------------------------------------------------/


\begin{document}%----------------------INICIO DOCUMENTO------------|
%------------------------------------------------------------------|
\input{Portada}

% \tableofcontents

\newpage

%%%%%%%%%%%%%%%%%%------------------------------------
%%%%%%%%%%%%%%%%------------------------------------
%%%%%%%%%%%%%%------------------------------------
\section{Liga del proyecto final}
%%%%%%%%%%%%%%------------------------------------
%%%%%%%%%%%%%%%%------------------------------------
%%%%%%%%%%%%%%%%%%------------------------------------


%\begin{small}
    \begin{verbatim}
        Liga del Proyecto 
    \end{verbatim}            
%\end{small}



%%%%%%%%%%%%%%%%%%------------------------------------
%%%%%%%%%%%%%%%%------------------------------------
%%%%%%%%%%%%%%------------------------------------
\section{Ramas del proyecto}
%%%%%%%%%%%%%%------------------------------------
%%%%%%%%%%%%%%%%------------------------------------
%%%%%%%%%%%%%%%%%%------------------------------------

La rama de entrega es:
\begin{itemize}
    \item[]  \texttt{proyecto-dwf2024-1}
    
                En esta rama subiremos la versión final del proyecto 

    \item[]  \texttt{proyecto/navbar}
    
                En esta rama haremos todos los cambios relacionados con Navbar, Sidebar y el Footer
\end{itemize}



%%%%%%%%%%%%%%%%------------------------------------
%%%%%%%%%%%%%%------------------------------------
\subsection{Estructura de Ramas}
%%%%%%%%%%%%%%------------------------------------
%%%%%%%%%%%%%%%%------------------------------------

La estructura de ramas se verá de este estilo
\begin{verbatim}
- proyecto-dwf2024-1
  |
   - proyecto/navbar
   - proyecto/<otras funciones>
\end{verbatim}

\begin{enumerate}
    \item Cada rama que se cree debe ser tomada de \texttt{proyecto-dwf2024-1} 
    \item El inicio debe ser \texttt{proyecto/}
    \item Después del \texttt{../}  deberá ir un nombre acorde a lo que se va a modificar, añadir o solucionar
\end{enumerate}

Es importante solo modificar los archivos necesarios para cada rama, de esta forma hacer las integraciones resultara más sencillo 

\end{document}